%\thispagestyle{empty}

A automação das atividades industriais tem como objetivo melhorar a eficiência de processos produtivos, reduzindo custos e aumentando a segurança. Em lavanderias industriais, a medição de nível de detergente líquido é um elemento fundamental para o gerenciamento de ativos, principalmente devido à necessidade de manter um fluxo contínuo dos processos de lavagem. Dessa forma, o trabalho apresenta uma solução implementada nos reservatórios da lavanderia industrial da Santa Casa da Misericórdia de Bragança, em Portugal, usando uma abordagem de internet das coisas, na qual integra um sistema de medição com conexão Wi-Fi, capaz de monitorar e registrar o nível de detergente líquido dos reservatórios em tempo real. Com isso, foi desenvolvido um sistema microcontrolado responsável por realizar as medições de nível ulilizando sensor ultrasônico, na qual os dados são enviados para um banco de dados e, através de uma plataforma web, o cliente consiga acessar de forma remota o resultado das medições. Para facilitar a instalação do sistema nos reservatórios, um bujão foi desenhado sob medida e impresso em 3D.

\bigskip

\noindent {\textbf{Palavras-chave:} Internet das Coisas; Banco de Dados; Desenvolvimento web; Sistema de Medição de Nível Ultrassônico;}
