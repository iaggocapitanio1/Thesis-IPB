\chapter{Conclusions and Future Work}\label{cap:conclusions}

The concepts of the Industry 4.0 comprises the idea of making the production more flexible, by using cutting edge technologies in order to maintain the competitive business environment. In the field of laundry industry, the necessity to ensure an effective logistics is essential to keep the quality and productivity of the service. In this context, the present work developed a solution based on \gls{IoT} to be applied in industrial washing machines. The developed integrated system demonstrated to be capable of monitoring and recording the detergent liquid level in a web database, allowing the laundries management to access the measurement results through a web platform.

To implement the system in the reservoirs, the printed device model evidenced to be efficient, as it provides a good positioning, for the ultrasonic sensor, to take the readings from the detergent liquid surface. Furthermore, it performs a very simply and compact solution, as it allows the measurements to be collected without being necessary to make modifications on the reservoir's structure, making it an attractive differential for the adopted solution. The analysis results from the measurements indicated that the use of low-cost ultrasonic sensor is very appropriate for this application, as the failed readings could be treaties, with the use of an embedded system, providing very good accuracy of the measurements. Also, the developed device proved promising to be used for reservoirs with heights up to 400 cm. For the web database connection, the use of \gls{PHP} \gls{API}s demonstrated to be efficient, being able to read large amounts of data from the database without compromising the user experience, however, security threats may be a problem due to the none use of authentication and lack of encryption. Finally, the use of \gls{AJAX} technique provides a wealth combination with the \gls{IoT} solution adopted, enabling the Web document to be dynamically displayed through the \gls{DHTML} web pages.

In general, there are several ways to perform level measurements and it represents an important indicator for process control. The use of direct methods may be simpler than the indirect ones, however, it may not be suitable for automated control. The non-contact level measurement using ultrasonic sensor, embedded alongside with a microcontroller, demonstrated to be affordable as a part of \gls{IoT} solution for industrial washing machines. Besides the detergent liquid level, the solution developed here may be applied for others liquids, since it does not have dynamic changes.

For the future works, it is expected the system to be implemented, in the laundry facilities of the \gls{SCMB}, with the objective to be tested others parameters, especially regarding the durability of the developed device, i.e., to check if the environment is damaging, in the course of time, the electronic components and/or the PLA material. Moreover, another important criterion is the implementation of \gls{API} protection, such as REST and SOAP, which must be developed for the work advancement.